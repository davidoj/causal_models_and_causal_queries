
% \section{Philosophical Preliminaries}

% Causality is an imprecise concept for which various mathematical formalisations have been proposed. There is ongoing debate as to whether the mathematical formalisations adequately capture the concept. The approach here is to agnostic on the nature of causes and instead focuses on the efficacy of the mathematical formalisms in addressing particular types of query. We will call the formalisms causal models and the queries causal queries, but our relationship to the metaphysics of causality is as an inspiration for potentially useful methods rather than a true form we seek to emulate.

\section{Graphical Models}


\begin{definition}[Directed Acyclic Graph]
A Directed Acyclic Graph (DAG) $G$ is a set of vertices and directed edges $(V,E)$. A directed edge $E$ is an ordered pair of vertices $V_1$ and $V_2$, is represented by $V_1 \to V_2$ and identifies $V_1$ as a parent of $V_2$ and $V_2$ as a child of $V_1$. We say the head of $E$ points to $V_2$ and the tail points to $V_1$.
\end{definition}

\begin{definition}[Paths and Directed Paths:]
A path $p$ consists of a sequence of nodes $\mathcal{V}=\{V_i:1\leq i \leq n\}$, $n\in\mathbb{N}$ along with a sequence of edges $\mathcal{E}=\{E_i:1\leq i \leq n-1\}$. The edge $E_i$ connects nodes $V_i$ and $V_{i+1}$, no edge appears in $\mathcal{E}$ more than once and no vertex appears in $\mathcal{V}$ more than once.

A directed path is a path from $V_i$ to $V_j$ such that every edge $E_k$ is oriented from $V_k$ to $V_{k+1}$.
\end{definition}

\begin{definition}[Descendants and Ancestors]\label{def:descendants_and_ancestors}
A descendant of $V_i$ in $G$ is a child of $V_i$ or the child of a descendant of $V_i$. Equivalently, $V_j$ is a descendant of $V_i$ iff there is a directed path from $V_i$ to $V_j$.

An ancestor of $V_i$ in $G$ is a parent of $V_i$ or a parent of an ancestor of $V_i$. Equivalently, $V_j$ is an ancestor of $V_i$ iff there is a directed path from $V_j$ to $V_i$.
\end{definition}

\begin{definition}[Head-to-head node]
A head to head or collider node $V_{h}$ in a path $p$ is a node where the two adjacent edges $E_i$ and $E_{i+1}$ in $p$ are oriented as $V_{h-1}\rightarrow V_h$ and $V_h\leftarrow V_{h+1}$ respectively.
\end{definition}

\begin{definition}[Blocked path]
A path $p$ is blocked by a set $S\subset V$ iff $p$ contains a node that is not a head-to-head node that is also in $S$, or if $p$ contains a head-to-head node which is itself in $S$ or has a descendent in $S$.
\end{definition}

\begin{definition}[d-separation]\label{def:dsep}
Two sets of nodes $A,B\subset V$ are d-separated in $G$ by $S\subset V$ iff every path starting with a node $A_i\in A$ and ending with a node $B_i\in B$ is blocked by $S$. We write this relation $A \perp_G B | S$.
\end{definition}

\section{Probability distributions, marginals and conditionals}

All definitions are for discrete probability spaces. I will extend them to continuous spaces when it becomes necessary.

\begin{definition}[Probability Space]
A probability space is a triple $(\Omega,\mathcal{F},P)$, where $\Omega$ is an arbitrary sample space, $\mathcal{F}$ is a $\sigma$-algebra on $\Omega$ and $P:\mathcal{F}\to[0,1]$ satisfies the properties of a measure over $\mathcal{F}$ as well as $P(\Omega)=1$.
\end{definition}

\begin{definition}[Random variable]\label{def:random_variable}
Given a probability space $(\Omega,\mathcal{F},P)$ and a measurable space $(E,\mathcal{E})$, a random variable is a measurable function $X:\Omega\to E$.

Abusing notation somewhat, we will write $P(X\in B)$ to mean $P(X^{-1}(B))$ for $B\in\mathcal{E}$. 

We will write the $\sigma$-algebra associated with a random variable $X$ as $\sigma(X)$. We will also sometimes discuss the range of values $X$ may take, denoted by $\text{Range}(X)$.
\end{definition}

\begin{definition}[Probability distribution]
Given a random variable $X$ associated with a probability space $(\Omega,\mathcal{F},P)$, the probability distribution of $X$ is the map $f:\sigma(X)\to [0,1]$ such that $f(B)=P(X\in B)$ for $B\in\sigma(X)$. We will write the probability distribution over $X$ as $P(X)$.

For random variables $X$, $Y$ associated with measure spaces $(D,\mathcal{D})$ and $(E,\mathcal{E})$, $P((X\in B)\wedge (Y \in C))$ for $B\in\mathcal{D}$, $C\in\mathcal{E}$ means $P(X^{-1}(B)\cap Y^{-1}(C))$, and we write the distribution over $X$ and $Y$ as $P(X\wedge Y)$.

\end{definition}

For the following definitions, assume we have a set of random variables $V$ and a probability space with measure $P$:

\begin{definition}[Conditional probability]
For a finite probability space, the conditional probability $P(V\setminus S|S\in s):\sigma(V\setminus S)\to [0,1]$ is \[P(V\setminus S|S\in s)=\frac{P(V)}{P(S\in s)}\]

For finite probability spaces, the conditional probability is undefined if $P(S\in s)=0$.

TBD: conditioning on probability 0 events in infinite spaces.
\end{definition}

\begin{definition}[Marginal probability]
The marginal probability of $A\subset V$, $P(A):\sigma(A)\to [0,1]$ is \[P(A):=\sum_{v\in \sigma(V\setminus A)} P(A \wedge V\setminus A \in v)\]
\end{definition}

\begin{definition}[Independence]
$A\subset V$ and $B\subset V$ are independent iff $P(A\wedge B)=P(A)P(B)$. We write independence $A\CI_{P(V)} B$.
\end{definition}

\begin{definition}[Conditional independence]
Given $A, B, S \subset V$, $A$ is conditionally independent of $B$ given $S$ iff $A\CI_{P(V\setminus S|S\in s)} B$ for all $s\in\sigma(S)$. We write this $A\CI_P B | S$.
\end{definition}

\section{Faithfulness}
\begin{remark}
Compatibility and faithfulness are both standard definitions, transparency is not. I chose transparency for the connotation that "the probability distribution doesn't mislead you with spurious independences".
\end{remark}

\begin{definition}[Compatibility]\label{def:compatibility}
A probability distribution $P$ is compatible with a DAG $G$ iff $A\perp_G B |S \Rightarrow A\CI_P B | S$ for all $A, B, S \subset V$.

Another term for compatibility is \emph{Markovian}
\end{definition}

\begin{definition}[Transparency]
A probability distribution $P$ is transparent with respect to a DAG $G$ iff $A\CI_P B|S \Rightarrow A\perp_G B|S$ for all $A,B,S\subset V$.
\end{definition}

\begin{definition}[Faithfulness]\label{def:faithfulness}
A probability distribution $P$ over variables $V$ is faithful to a DAG $G$ over vertices $V$ iff $A \CI_P B | S \Leftrightarrow A\perp_G B | S$ for all $A,B,S\subset V$.
\end{definition}

From these definitions, $\text{Transparency}+\text{Compatibility}\Leftrightarrow \text{Faithfulness}$. Any distribution $P$ is compatible with a fully connected graph and transparent with respect to a fully disconnected graph, but some distributions have no graph to which they are faithful.

A graph $G$ may not permit faithfulness of some probability distribution $P$ simply by including spurious arrows. For example, if $A\CI_P B | \varnothing$, the graph $G=(\{A,B\},A\to B)$ is compatible but not faithful to $P$, and we can create a faithful graph by removing the edge $A\to B$.

It is also possible, however, that a graph can fail to be faithful in a manner that cannot be resolved by removing an edge (see Example \ref{ex:linear_unfaithfulness}). The strictly weaker condition of \emph{minimality} captures the intuition of "a graph without superfluous arrows".

\begin{definition}[Minimality]\label{def:minimality}
A graphical model of a Markovian probability distribution over random variables $\mathbf{V}$ is minimal if, for every $V_i\in\mathbf{V}$ and every parent $p$ of $V_i$
\[V_i\not\CI p|\mathrm{Parents}(V_i)\setminus p\]
\end{definition}

\subsection{Types of unfaithfulness}

Shurz and Gerbharter detail the following types of unfaithfulness \cite{schurz_causality_2016}. It is not claimed to be an exhaustive typology, but it is helpful to discuss where different types of unfaithfulness can arise.

\begin{definition}[Cancellation unfaithfulness]

\end{definition}

\begin{definition}[Determinism unfaithfulness]

\end{definition}

\begin{definition}[Intransitiviey unfaithfulness]\label{def:intrans_unfaith}

\end{definition}

Ramsey offers the following exhaustive typology of faithfulness \cite{ramsey_adjacency-faithfulness_2012}. This breakdown is useful for constructing algorithms, as while adjacency faithfulness is not testable, orientation faithfulness is.

\begin{definition}[Adjacency faithfulness]

\end{definition}

\begin{definition}[Orientation faithfulness]

\end{definition}

\subsection{Directed Markov properties}

A probability distribution $P$ that is compatible with a DAG $G$ satisfies a number of equivalent properties in addition to the $d$-separation condition given in Defintion \ref{def:compatibility} (proofs omitted).

\begin{definition}[Markov factorisation]\label{def:markov_factorisation}
If $P$ is compatible with $G$, then we can write
\begin{align}
    P(V) = \prod_{V_i\in V} P(V_i|\mathrm{pa}_i)
\end{align}
Where $\mathrm{pa}_i$ is the set of parents of $V_i$ in $G$
\end{definition}

\begin{definition}[Local directed Markov property]\label{def:local_markov}
Denote by $\mathrm{nd}_i$ the set of all non-descendents $\mathrm{nd}_i=V\setminus (\mathrm{de}_i\cup\{V_i\})$.

Then, given $P$ compatible with $G$,
\begin{align}
    V_i \CI_P \mathrm{nd}_i | \mathrm{pa}_i
\end{align}
\end{definition}




\section{Additional Definitions}

\begin{definition}[Markov equivalence class]
The Markov equivalence class $\mathcal{M}_G$ of graph $G$ is the set of graphs which share d-separation properties with $G$: $\mathcal{M}_G=\{G'\in\mathcal{G}:A\perp_{G'}B | S \Leftrightarrow A\perp_{G}B|S\}$. In other words, given the set of distributions $\mathcal{P}_G:=\{P|P\text{ is faithful to }G\}$, the Markov equivalence class of $G$ is the set of all graphs to which every distribution in $\mathcal{P}_G$ is faithful.
\end{definition}

\begin{definition}[Markov superclass]
The Markov superclass $\mathcal{M}_G$ of graph $G$ is the set of graphs $\{G'|A\perp_{G'} B \implies A\perp_G\}$. In other words, the Markov superclass of $G$ is the set of all graphs compatible with probability distributions that are compatible with $G$. 
\end{definition}

\begin{definition}[Structural equation model]\label{def:structural_equation}
Given a set of random variables $\mathcal{X}=\{X_i$\}, define $\mathcal{X}_{<n}=\{X_i|i<n\}$. A structural equation model is a set of assignments for each $i\in\{1,...,n\}$ of the form

\begin{align*}
    X_n &:= f_n(Q_{<n},u_{X_n})
\end{align*}

Where $Q_{<n}\subset \mathcal{X}_{<n}$ and $u_{X_i}$ are random variables usually assumed to be jointly independent with associated distributions $\mathcal{L}(u_{X_i})$. 
\end{definition}

\begin{definition}[Corresponding graph]\label{def:sem_corresponding_dag}
The corresponding graph for a SEM is the DAG $G=(V,E)$ where $X_i\to X_j\in E$ iff $X_i$ appears on the right hand side of the assignment for $X_j$.
\end{definition}

\section{Faithfulness properties of Structural Equation Models}

\subsection{SEM + joint independence of noises $\Rightarrow$ compatibility}

Joint independence of the noises $E_{X_i}$ is sufficient to ensure that the probability distribution defined by such a SEM is compatible with a DAG constructed via Definition \ref{def:sem_corresponding_dag} \cite{spirtes_causation_1993} (pp41). The argument involves showing that conditioning on variables on the right hand side of an assignment in an SEM create independences, and that these independences map to d-separation in the related DAG.

\subsection{SEM + joint independence of noises $\not\Rightarrow$ transparency}

Here we show by way of example that a probability distribution defined by a SEM with jointly independent noises can fail to be faithful to a DAG constructed as above.

\begin{example}\label{ex:linear_unfaithfulness}
Consider the following assignments constituting a structural equation model over $\{X_1,X_2,X_3\}$
\begin{align*}
    X_1 &:= u_{X_1} \\
    X_2 &:= a X_1 + u_{X_2} \\
    X_3 &:= b X_1 + c X_2 + u_{X_3}
\end{align*}

Following Definition \ref{def:sem_corresponding_dag}, the corresponding DAG to this assignment is:

\begin {center}
\begin {tikzpicture}[-latex ,auto ,node distance =4 cm and 5cm ,on grid ,
semithick ,
state/.style ={ circle ,top color =white ,
draw , text=blue , minimum width =1 cm}]
\node[state] (A) [left] {$X_1$};
\node[state] (B) [right = of A] {$X_2$};
\node[state] (C) [below = of B] {$X_3$};
\path (A) edge [] node[above] {} (C);
\path (A) edge [] node[above] {} (B);
\path (B) edge [] node[above] {} (C);
\end{tikzpicture}
\end{center}

A probability distribution transparent with respect to this DAG would have no conditional independences, as this DAG features no d-separation. However, if $ac=-b$, the assignments above reduce to 

\begin{align*}
    X_1 &:= E_{X_1} \\
    X_2 &:= a X_1 + E_{X_2} \\
    X_3 &:= c E_{X_2} + E_{X_3}
\end{align*}

By the assumption of jointly independent noises, under this assignment we will have $X_3 \CI_P X_1 | \varnothing$, and so this model fails to be transparent with respect to the DAG illustrated above.
\end{example}

\section{Faithfulness as nongenericity}

Assuming faithfulness can be justified by the argument that, under certain conditions, faithfulness is only violated by probability distributions that are nongeneric.

Take the set of all DAGs $\mathcal{G}$ over vertices indexed by $i\in\{1,..,n\}$ and an arbitrary set of probability distributions $\mathcal{P}$ over random variables $V=\{V_i|i\in\{1,..,n\}\}$. Assume for each graph $G$ there is an associated measure $\mu_G$ over a $\sigma$-algebra on $\mathcal{P}$.

A contrast is a function $C:\mathcal{P}\to\mathbb{R}^n$.

A distribution $P$ is generic relative to $\mu_G$, $C$ if

\begin{align*}
    \mu_G(\{P':C(P')=C(P)\}) \neq 0
\end{align*}

Consider the contrast:
\[C_{FF}(P) = [\mathds{1}_{X\CI_P Y|S}|X,Y\in V,S\subset V]\]

Where $\mathds{1}_s$ is a function that evaluates to $1$ if $s$ is true and $0$ otherwise, and $\rho_{XY|S}$ is the partial correlation of $X$ and $Y$ controlling for $S$.

For each graph $G$ we can construct a vector 
\[\text{Sep}(G) = [\mathds{1}_{X\perp_G Y|S}|X,Y\in V, S\subset V]\]

The vectors $C_{FF}(P)$ and $\text{Sep}(G)$ represent the conditional independence properties of $P$ and the d-separation properties of $G$ respectively. They are defined such that $C_{FF}(P)=\text{Sep}(G)\Leftrightarrow P\text{ is faithful to }G$ (this is a straightforward consequence of Definition \ref{def:faithfulness}.

Under some additional assumptions, the $\{P|C_{FF}(P)=\text{Sep}(G)\}$ also characterises the set of distributions $\mathbf{P}_G\subset \mathcal{P}$ which are generic with respect to $C_{FF}$ and the measure $\mu_G$. Examples of these assumptions are given below, with proofs given in the referenced literature.

Take $\mathcal{P}$ to be the space of multivariate Gaussians with parametrisation $\mathbf{\theta}\subset \mathbb{R}^{|V|^2}$ and each $\mu_G$ a probability measure such that, for any set of parametrisations $\Theta\subset \mathbb{R}^{|V|^2}$ with Lebesgue measure zero, the measure of the associated set of distributions $\mu_G(\mathbf{P}^\Theta)=0$. It has been shown that, given these assumptions, all distributions in $\mathbf{P}_G$ are generic with respect to $C_{FF}$ and $\mu_G$, while any distributions not in $\mathbf{P}$ are nongeneric \cite{spirtes_causation_1993} (pp41-2). It has also been shown that an analogous result holds for discrete distributions \cite{meek_strong_2013}.

In the infinite sample case, these arguments motivate the use of the faithfulness assumption:
\begin{enumerate}
    \item If we accept that the class of measures $\mu_G$ that respect the assumption regarding a density on $\theta$ contains the probability distribution of cause-effect relationships in nature, then assuming faithfulness will reject a correct graph with probability 0
    \item The faithfulness assumption allows us to reject all graphs outside of a single Markov equivalence class, which will usually reduce the set of causal models that we hypothesise may generate the data
\end{enumerate}

Regarding the "usually" on the second point above, if the Markov equivalence class accepted is fully connected, then there is no restriction on the probability distribution implied. On the other hand, if faithfulness is assumed there may be a restriction on possible interventional distributions (i.e. distributions under $do()$ interventions) - I don't know at this point.

All these points have been discussed in the literature, bar the formulation in terms of a contrast and the final comment about whether fully connected graphs imply restrictions on interventional distributions if faithfulness is assumed.

\begin{remark}
A corollary of this is that a probability distribution $P$ that is faithful to a graph $G$ is not only generic with respect to $C_{FF}$ and $\mu_G$, but also nongeneric with respect to $C_{FF}$ and $\mu_{G'}$ for any $G'$ not in the Markov equivalence class of $G$.
\end{remark}

% Questions I'm interested in pursuing:
% \begin{itemize}
%     \item What is the appropriate way to formulate the two items in the list above in a more rigorously? Is it possible to produce "performance metrics" for causal discovery rules based on these items?
%     \item Assuming a measure of performance and dropping the infinite sample assumption, can we retain a general class of priors $\mu_G$ that will guarantee reasonable performance of causal discovery based on the faithfulness assumption? 
%     \item Can the two questions above be applied to other causal discovery rules?
% \end{itemize}


\section{Causal models}

In this section I propose a general definition of causal models and fit various existing definitions into this one.

\begin{definition}[Causal model]\label{def:causal_models}
Given a set $\mathbf{V}$ of random variables and a set $\mathcal{I}$ of interventions, a causal model is a map $M:\mathcal{I}\to \mathcal{P}(\mathbf{V})$ where $\mathcal{P}(\mathbf{V})$ represents the set of all probability distributions over $\mathbf{V}$.

The probability distribution $M(I)$ behaves for most purposes like the conditional $P(V|I)$, however the joint $P(V,I)$ does not necessarily exist. I will maintain the former notation for the time being for this reason.

To reflect notation used in other literature, I will sometimes use $P_I$ to represent the probability distribution $M(I)$.
\end{definition}

\begin{remark}
A causal model so defined is similar to a conditional distribution. There is an intuitive sense in which $P(V|I)$ means the same thing as $M(I)$. However, unlike the conditional $P(V|I)$, a causal model doesn't require the joint distribution $P(V,I)$ to be well defined.
\end{remark}

\subsection{Cause and effect models}

A particular class of causal model is built on the notion of cause-effect relationships. A cause-effect relationship is a directed binary relation. For a set of variables $V$, we also require that the set of causal relations between them doesn't induce any cycles. It may be the case that for some systems of interest there are no acyclic cause-effect assignments that yield a reasonable causal model; for such systems, we will have to consider a broader class of causal models. 

Given the assumptions above, the cause-effect relationships of a cause-effect model form a directed acyclic graph (DAG) over the model's variables. One fact to bear in mind in the following discussion is that there are two types of DAGs that are often discussed in relation to causal inference. The first, sometimes termed a Bayesian network, is a DAG $G=(V,E)$ which is compatibile with the probability distribution $P$ over variables we identify with the nodes $V$ (see Definition \ref{def:compatibility}). Sometimes, a Bayesian network might additionally be supposed to be minimal or faithful to $P$ (see Definitions \ref{def:minimality} and \ref{def:faithfulness}). In all cases, there are usually several graphs $G$ which bear the appropriate relations to $P$.

The second, which we will call a causal graph, is related to a full causal model $M$ rather than a single probability distribution $P$. Under some assumptions, the causal graph for a model $M$ is unique (see \ref{th:unique-minimal-graph}). 

\begin{definition}[Direct causes and effects]
A cause-effect relationship is a directed binary relation $\rightsquigarrow$. Given $V_i\rightsquigarrow V_j$, we say $V_i$ is a direct cause of $V_j$ and $V_j$ is a direct effect of $V_i$.

For a set of variables $\mathbf{V}$, cause-effect relationships $E$ and $V_i\in\mathbf{V}$, write  the set of direct causes of $V_i$ as $\mathrm{dc}_i=\{V_j|(V_j\rightsquigarrow V_i)\in E\}$ and the set of direct effects as $\mathrm{de}_i=\{V_j|(V_i\rightsquigarrow V_j)\in E\}$.
\end{definition}

\begin{definition}[Causes and effects]
A set of cause-effect relationships $E$ over variables $V$ induces a directed graph $G=(V,E)$. 

A cause of a variable $V_i$ is an ancestor of $V_i$ in $G$, and an effect of $V_i$ is a descendant of $V_i$. See Definition \ref{def:descendants_and_ancestors}.

Write the set of causes of $V_i$ as $\mathbf{c}_i$ and the set of effects as $\mathbf{e}_i$.

\end{definition}

\begin{definition}[Noneffects]
Define the noneffects of $V_i\in\mathbf{V}$ as $\mathrm{ne}_i:= (\mathbf{e}_i\cup \{V_i\})^C$.
\end{definition}

\begin{definition}[Causal Markov Condition (CMC)]
A probability distribution $P$ is Markov with respect to a cause-effect graph $G=(\mathbf{V},E)$ if it satisfies any of the equivalent conditions of compatibility (Def \ref{def:compatibility}), Markov factorisation (Def \ref{def:markov_factorisation}) or the local Markov property (Def \ref{def:local_markov}).

The CMC is typically stated in terms of the local Markov property:
\begin{align}
    V_i\CI_P \mathrm{ne}_i | \mathrm{dc}_i
\end{align}
\end{definition}

Along with the assumption of acyclicity, CMC is equivalent to the property that the probability distribution $P$ is consistent with $G$ (i.e. d-separation in $G$ implies conditional independence in $P$). 


\begin{definition}[Markovian cause effect model]
A causal model $M$ along with a directed acyclic graph $G$ for which every $P\in \mathrm{Range}(M)$ is Markovian with respect to $G$ is a Markovian causal model.
\end{definition}

\subsection{Causal Bayesian networks}

Here I will define causal Bayesian networks (CBNs), a particular case of Markovian cause effect models. These are Bayesian networks with the "do" operator of the type discussed by Pearl.

A CBN $M$ is defined via the operation of $M$ on allowable interventions $I$. We will first define the notation and spaces for these interventions, and then the restrictions on the operation of $M$ required by a CBN.

\begin{definition}[Atomic intervention (CBN)]
Given a CBN $M$ over a set of random variables $V$, an atomic intervention is an index $i$ along with an element of the range of the corresponding variable $V_i$. Given $i\in\{0,..,|V|\}$, the atomic intervention $I_i \in \{i\}\times \text{Range}(V_i):=\mathcal{I}_{i}$. 

We can write atomic interventions in two ways $I:=[i,v_i]\equiv do(V_i=v_i)$.

% Given an atomic interventions $I_i \in \mathcal{I}_{i}$, a CBN $M$ has the following property for some probability distribution $Q$:
% \begin{align}
%         I_{i} = [i,v_i] \implies M(I_{i}) = \delta(V_i-v_i)Q(V\setminus \{V_i\})
% \end{align}
\end{definition}

\begin{definition}[Intervention (CBN)]
An intervention $I_U$ on a set of variables $V$ is defined as the union over any set of atomic interventions on the variables of $V$: 
\[I_U=\cup_{i\in U\subset \{1,..,V\}} I_{i}\]

Define the intervention space $\mathcal{I}_U:=\prod_{i\in U} \mathcal{I}_i$
\end{definition}

\begin{definition}[Passive intervention (CBN)]
Define the passive intervention $I_0:=\varnothing$.
\end{definition}

\begin{definition}[Intervention space (CBN)]\label{def:cbn_intervention_space}
The space of permissible interventions for a CBN is $\mathcal{I} := I_0\cup \left(\cup_{U\in\mathds{P}(\{1,..,|V|\})}\right)$
\end{definition}

\begin{definition}[Target variables]
For an intervention $I_U = \cup_{i\in U} I_{i}$, we call the set $V_U = \{V_i|i\in U\}$ the target variables.
\end{definition}

\begin{definition}[Causal Bayesian Network]\label{def:CBN}
Consider the set of interventions $\mathcal{I}$ and a causal model $M:\mathcal{I}\to\mathcal{P}$ over random variables $V$, and write $P_A=M(I_A)$. $M$ is a Causal Bayesian Network if
\begin{enumerate}
    \item There is some cause-effect graph $G=(V,E)$ such that $M$ is Markovian relative to $G$ and
    \item For intervention $I_U$ and any $V_i \in \{V_k|k\in U\}^C$, $P_U(\mathrm{dc}_i) > 0 \implies P_U(V_i|\mathrm{dc}_i) = P_0(V_i|\mathrm{dc}_i)$ where $\mathrm{dc}_i$ is defined relative to $G$
    \item $I_U = \cup_{i\in U}[i,v_i] \implies P_U(\{V_i|i\in U\}) = \prod_{i\in U} \delta_{V_iv_i}$
\end{enumerate}
Here $\delta_{ij}$ is the Kronecker delta function, defined such that $\delta_ij=0$ for $i\neq j$ and $\delta_{ii} = 1$.

Call any graph $G$ for which properties 1 \& 2 hold the \emph{associated graph}.
\end{definition}

\begin{definition}[Truncated factorisation]
A causal model $M$ obeys truncated factorisation relative to a graph $G$ if, for every $I_U=\cup_{i\in U} [i,v_i]$, the distribution $P_U$ can be written as product of the conditionals of non-target variables under the ``null intervention'' distribution $P_0$ and delta functions over the target variables:
\[P_U(\mathbf{V}) = \prod_{i\in U} \delta (V_i-v_i)\prod_{i\not\in U} P_0(V_i|\mathrm{dc}_i) \]
\end{definition}

\begin{theorem}[CBN satisfies truncated factorisation]
A CBN has the truncated factorisation property.
\end{theorem}

\begin{proof}
$M(I_U)$ is Markov with respect to an associated graph $G$ for any $I_U$, and so it is possible to write
\begin{align}
    P_U(\mathbf{V})  &=  \prod_{i\in\{1,..,|V|\}} P_U(V_i|\mathrm{dc}_i) \nonumber \\
    &=  \prod_{i\in U} P_U(V_i|\mathrm{dc}_i) \prod_{i\not \in U} P_U(V_i|\mathrm{dc}_i) \label{eq:markov_factorisation_split}
\end{align}
with $\mathrm{dc}$ defined relative to $G$.

By property 3 of a CBN, $I_U =\cup_{i\in U}[i,v_i] \implies P_U(V_i=v_i)=1$ and $P_U(V_i\neq v_i)=0$ for all $i\in U$. Noting that $P(A)=0$ implies $P(A|X)=0$ and $P(A)=1$ implies $P(A|X)=1$ for any $X$ of nonzero probability, so $P(A)=\delta_{Aa}\implies P(A,B)=\delta_{Aa}P(B)$. Direct substitution into Eq. \ref{eq:markov_factorisation_split} yields

\[P_I(\mathbf{V}) = \prod_{i\in U} \delta(V_i-v_i)\prod_{i\not\in U} P_U(V_i|\mathrm{dc}_i) \]

Applying property 2, we have

\[P_I(\mathbf{V}) = \prod_{i\in U} \delta(V_i-v_i)\prod_{i\not\in U} P(V_i|\mathrm{dc}_i) \]

\end{proof}

\begin{definition}[Graphical truncation]
Given an intervention space $\mathcal{I}$ and the set of all directed acyclic graphs $\mathcal{G}$ over nodes indexed by $i\in\{1,..,n\}$, graphical truncation is the operation $T:\mathcal{I}\times\mathcal{G}\to\mathcal{G}$ that takes a graph $G=(V,E)$ and intervention $I_U$, and returns the graph $G_U=(V,E_U)$ where $E_U=\{(V_i\to V_j)|(V_i\to V_j)\in E\text{ and }j\not\in U\}$.
\end{definition}

\begin{corollary}[CBN satisfies graphical truncation]
A CBN $M$ with an associated graph $G$ has the property that for every $I_U\in\mathcal{I}$, $P_U$ is Markovian relative to the truncated DAG $G_U:=T(I_U,G)$.
\end{corollary}

The required conditional independences are a direct result of the truncated factorisation property.

\begin{remark}
Conditions 1 and 2 are not on their own sufficient for graphical truncation.

Example: Consider a two variable Markovian causal model $M$ with $G=(\{X,Y\},\{X\rightsquigarrow Y\})$. Chose some distribution $P_0(X,Y)$ for which $X\not\CI_{P_0} Y| \varnothing$. Clearly, $P(X,Y)$ is Markov with respect to $G$. Consider an intervention $I$ which targets $Y$ such that $M(I)=P_I(X,Y)=P_0(X,Y)$. $P_I(X,Y)$ is trivially also Markov with respect to $G$, and $P_I(X)=P(X)$, however $X\not\CI_{P_I} Y| \varnothing$.
\end{remark}

\begin{remark}
It is straightforward to show that truncated factorisation implies conditions 2 and 3, so an alternative definition of CBNs would be to replace conditions 2 and 3 with truncated factorisation instead.

It is also possible to define weaker truncated factorisation schemes, for example we could require 
\begin{align}
    P_U(\mathbf{V})  &=  \prod_{i\in U} P_U(V_i) \prod_{i\not \in U} P(V_i|\mathrm{dc}_i)
\end{align}
that will respect graphical truncation, or
\begin{align}
    P_U(\mathbf{V})  &=  \prod_{i\in U} P_U(V_i|\mathrm{dc}_i) \prod_{i\not \in U} P(V_i|\mathrm{dc}_i)
\end{align}
that will not.
\end{remark}

As mentioned earlier, while a Bayesian network usually does not have a unique graph, even if we require minimality or faithfulness. A causal Bayesian network, on the other hand, has a unique minimal associated graph.

\begin{lemma}[$G$ is minimal $\implies T(I,G)$ is minimal]\label{lem:universal_minimality}
Given a CBN $M$ and an associated graph $G$ that is minimal with respect to $P_0$, $G_U$ is minimal with respect to $P_U$ for all interventions $I_U$.
\end{lemma}

\begin{proof}
We will proceed by showing $G_U$ is not minimal with respect to $P_U \implies G$ is not minimal with respect to $P_0$.

Suppose $G_U$ is not minimal with respect to $P_U$ for some $I_U\in\mathcal{I}$. Then there is some node $V_i\in \mathbf{V}$ with parent $p$ such that $V_i \CI_{P_U} p | \mathrm{dc}_i\setminus \{p\}$. Thus $P_U(V_i|\mathrm{dc}_i) = P_U(V_i|\mathrm{dc}_i\setminus \{p\})$. If $i\in U$, the parent set of $V_i$ would be empty in $G_U$, so we can exclude this possibility. If $i\not \in U$, we have $P_U(V_i|\mathrm{dc}_i)=P_0(V_i|\mathrm{dc}_i)$, and so $V_i\CI_{P_0} p|\mathrm{dc}_i\setminus \{p\}$. Thus $G$ is not minimal with respect to $P_0$.
\end{proof}

\begin{lemma}[Reduction to minimal graph]\label{lem:reduction_minimal}
Given a CBN $M$ and an associateg graph $G$, a minimal graph $G^*$ can be generated by removing edges from $G$.
\end{lemma}

\begin{proof}
If a graph is not minimal with respect to $P_0$, it is possible to identify a node $V_i$ with a parent $p$ such that \[V_i\CI_{P_0} p|\mathrm{dc}_i\setminus \{p\}\]

Consider the graph $G'$ created by removing the edge $p\to V_i$ from $G$. We will show that $M(I)$ satisfies the local Markov property with respect to $G'$ for all interventions $I$.

First, by the local Markov property on $P_U=M(I_U)$ and $G$, we have $V_i \CI_{P_U} \mathrm{ne}_i|\mathrm{dc}_i$. In addition, by assumption, $V_i\CI_{P_0} p|\mathrm{dc}_i\setminus \{p\}$. The contraction property of conditional independence then gives us $V_i \CI_{P_U} \mathrm{ne}_i\cup\{p\}|\mathrm{dc}_i\setminus\{p\}$. Noting that $\mathrm{ne}_i\cup\{p\}$ is the set of noneffects of $V_i$ in $G'$ and $\mathrm{dc}_i\setminus \{p\}$ is the set of direct causes in $G'$, we have verified the local Markov property as required.
\end{proof}

\begin{theorem}[A CBN has a unique minimal associated graph]\label{th:unique-minimal-graph}
Given a CBN $M$ with an associated graph $G$, there is a unique graph $G^*$ that is minimal with respect to $P_0$ and for which $T(I_U,G^*)$ is minimal with respect to $P_U$.
\end{theorem}

\textbf{Proof:} Construct a minimal graph $G^*$ from $G$ by removing any edges necessary. By Lemmas \ref{lem:universal_minimality} and \ref{lem:reduction_minimal}, $G^*$ produces minimal graphs compatible with $P_U$ under graphical truncation with $I_U$.

Suppose there is a graph $G'$ that disagrees with $G^*$ on at least one edge that also generates a set of graphs that render $M$ Markovian.

Suppose that an edge disagreement is between nodes $V_i$ and $V_j$. We will identify three cases:
\begin{enumerate}
    \item There is an edge between $V_i\to V_j$ in $G^*$, and no edge in $G'$
    \item There is no edge between $V_i$ and $V_j$ in $G^*$, and there is an edge $V_i\to V_j$ in $G'$
    \item There is an edge $V_i \to V_j$ in $G^*$, and an edge $V_j\to V_i$ in $G'$
\end{enumerate}

In case (1), consider $I_{\mathrm{ce}_j^{G^*}}$, an intervention on the direct causes and effects of $V_j$ with respect to $G^*$.  If $G'_{\mathrm{ce}_j^{G^*}}$ were Markov with respect to $P_{\mathrm{ce}_j^{G^*}}$, then we would have $V_i \CI_{P_{{\mathrm{ce}_j^{G^*}}}} V_j | \mathrm{dc}^{G^*}_{V_j}\setminus \{V_i\}$. This would imply that $G^*_{\mathrm{ce}_j^{G^*}}$ is not minimal with respect to $P_{\mathrm{ce}_j^{G^*}}$, contradicting Lemma \ref{lem:universal_minimality}.

In case (2), either $V_i$ is a descendent of $V_j$ in $G^*$, $V_j$ is a descendent of $V_i$ or neither are descended from the other. 

First, suppose $V_j$ is a descendent of $V_i$ in $G^*$. Then, by the local Markov property, $V_i\CI_{P_0} V_j | \mathbf{V}^{pa}_j$, as $V_j$ must be a nondescendent of $V_i$ by acyclicity. Then $V_i\to V_j$ in $G'$ violates minimality of $G'$. 

Next, suppose $V_i$ is a descendent of $V_j$ in $G^*$. Because $G'$ is acyclic, $V_i$ cannot be a descendent of $V_j$ in $G'$. Thus we can identify two nodes $V_k$ and $V_l$ such that $V_k\to V_l$ in $G^*$ and either $V_l\to V_k$ in $G'$ or no edge between $V_k$ and $V_l$ in $G'$, which reduces to case (3) and case (1) respectively.

In case (3), consider $G''$ which identical to $G'$ except it has no edge between $V_i$ and $V_j$. Under intervention on $V_i$, $G'_{V_i}$ and $G''_{V_i}$ are identical, as the intervention deletes the incoming edge on $V_i$ in $G'$. In particular, $G''_{pc_j}=G'_{pc_j}$, and so the argument for case (1) holds. $\square$

\text{Comment (minimality is necessary):} If we drop the requirement of minimality, choosing $G$ to be fully connected and dropping the graphical truncation rule will yield a graph that is Markov with respect to every $P_I\in\mathrm{Range}(M)$

\subsection{Other types of causal models}

\subsubsection{Structural equation models}

See Definition \ref{def:structural_equation} for a standard definition of structural equation models. 

\begin{definition}[Assignment (SEM - notation)]
Given an SEM $\mathcal{S}$ over variables $V$, $A_i$ denotes the assignment statement with variable $V_i$ on the left.

The SEM can thus be written $\mathcal{S}=\{A_i|i\in\{1,..,|V|\}\}$.
\end{definition}

I suggest the following definition for the causal model associated with a structural equation model:

\begin{definition}[Causal Model (SEM)]
Consider an intervention space $\mathcal{I}$ defined as in \ref{def:cbn_intervention_space}, the probability space $\mathcal{P}$ defined as the set of probability distributions on $V$ that are able to be written as an SEM and a causal model $M:\mathcal{I}\to\mathcal{P}$.

Write $\mathcal{S}_U=M(I_U)$ and $A_i^U\in\mathcal{S}_U$ denotes an assignment statement in $\mathcal{S}_U$.

A causal structural equation model $\mathcal{S}_U$ has the following properties for every intervention $I_U$:
\begin{enumerate}
    \item $I_U=\{[V_i,v_i]\}\implies A_i^U=(V_i:=v_i)$
    \item $A_i^U=A_i^0$ for $i\not\in U$
\end{enumerate}
Recall from earlier definitions we also require $\mathcal{S}_U$ to have jointly independent noises and acyclic variable assignments.

\end{definition}

Structural causal models so defined are CBNs \cite{pearl_causality:_2009}, but I don't know if there are CBNs that can't be written as structural causal models.

One feature of structural causal models is the possibility to derive deterministic maps by substituting constants for the noise variables $u_i$. This could be seen as a generalisation of interventions from operations that replace the right hand side of an assignment with a constant to an operation that makes a more general modification to the right hand side of an assignment.


\subsubsection{Generalisation of CBNs}

\textbf{CBN equivalence classes} In the context of \emph{causal discovery} (to be defined later), practitioners often deal with equivalence classes of CBNs under some equivalence relation $\sim$. A notable equivalence class is represented by \emph{completed partial directed acyclic graphs} (CPDAGS), which represent classes of CBNs that are faithful to a given set of conditional independences of $P_0$. CPDAGS are a type of chain graph, which combines directed and undirected edges \cite{spirtes_causation_1993}\cite{chickering_optimal_2003}.

\textbf{Marginal models} are the class of models that can be produced by marginalising over variables in a CBN. They are used to model systems with hidden variables. 

A notable class of marginal models is \emph{nested Markov models}, which characterise the equality constraints of marginal CBNs \cite{evans_margins_2015}. 

Marginal models can also imply inequality constraints on a probability distribution \cite{kang_inequality_2012}.

I'm not particularly familiar with this literature, and I have some very simple questions about it. For example:

\begin{itemize}
    \item Are there conditions analogous to compatibility and truthfulness for probability distributions in relation to marginal models?
    \item Are there rules for how marginal models, which model observational probability distributions, correspond to causal models akin to graphical truncation/truncated factorisation for CBNs?
\end{itemize}

\textbf{Limited intervention sets} investigate the characterisation of CBNs where the set of interventions is smaller than the complete intervention space $\mathcal{I}$ introduced in \ref{eq:intervention_space} \cite{hauser_characterization_2012}.

\textbf{General interventions} drop condition (2) of CBN model - that is, that interventions must set intervened variables to a particular value. They retain the parental adjustment and local Markov properties. Generalised interventions do not support the graphical truncation/truncated factorisation rules \cite{yang_characterizing_2018}.

\subsection*{Representing a causal model with an extended graph}

A graphical model or Bayesian network usually represents a single probability distribution, and a causal model is not just a single probability distribution. It is possible to explicitly represent "intervention" nodes in a graphical model, however.

\textbf{Intervention variable:} An intervention variable $I_A$ is a cause of each variable in some set of variables $A\subset V$, and has no causes of its own. $I_A\in \{0,1\}\times \mathrm{Range}(A)$. If $I_A=(0,\cdot)$ it is "passive" and $A$ is influenced by whatever its usual causes are. If $I_A=(1,a)$, $A$ is forced to the value $a$, regardless of anything else.

Consider a causal model $M$ over variables $\mathbf{V}$ which is Markovian with respect to a graph $G$ that does not involve interventional variables, and a variant model $M'$ that is Markovian with respect to an expanded graph $G'$ that features an interventional variable $I_A$ for each $A\subset\mathscr{P}(\mathbf{V})$. $M'$ has its domain restricted to probability distributions over interventional variables.

For each target distribution $P^*_{A}(A)$ we can associate a unique distribution over interventional variables $P^*_{I_A}(I_A)$ that induces the same distribution on $A$.

\textbf{Conjecture:} For every $M$ so defined, there is a model $M'$ so defined such that $M(P^*_A(A))=M'(P^*_{I_A}(I_A)$ for all $P^*_A$.
