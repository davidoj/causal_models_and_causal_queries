\begin{definition}[Pseudo-Random Variable Kernel]
    Given a random variable $X:E\to F$, a Markov kernel $K_X$ where, for $e\in E$ and $B\in\mathcal{F}$,
    \begin{align}
        K_X(e,B) = \mathds{1}_{X(e)\in B}
    \end{align}
    Will be called a Pseudo-Random Variable Kernel on $X$.
\end{definition}

\begin{lemma}[Pseudo-RV transform]
Given a random variable $X:E\to F$ and a pseudo-RV kernel $K_X$ and a measure $\mu$ on $(E,\mathcal{E})$, the image measure $\mu\circ X^{-1}$ is the same as the kernel transform $\mu K_X$.
\end{lemma}

\begin{proof}
For all $B\in\mathcal{F}$,
\begin{align}
    \mu K_X(B) &= \int_E \mu(de)K(e,B) \\
               &= \int_E \mu(de) \mathds{1}_{X(e)\in B} \\
               &= \mu(X^{-1} B)
\end{align}
\end{proof}

Pseudo-RV kernels are a convenience to allow for random variables to be used with graphical kernel composition without defining new rules for the inclusion of random variables.

\begin{table}[h]
    \centering
    \begin{tabular}{c|c|c}
        Name & Symbol & Type signature  \\
        \hline
        Observation at level $i$ & $\theta_i$ & $H\to \Delta(\mathcal{A})$ \\
        Policy choice & $\pi$ & $H\to \Delta(\mathcal{I})$ \\
        Resolution & $\mu$ & $H\times I \to \Delta(\mathcal{V})$  \\
    \end{tabular}
    \caption{Markov kernel definitions.}
    \label{tab:kernels}
\end{table}


