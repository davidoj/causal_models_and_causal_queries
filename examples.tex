$\mathscr{M}$ is the collection of kernels $\theta,\phi,\mu$. $Q$ is a sub-$\sigma$-algebra of $\mathcal{H}$ generated by $\phi$; i.e. it is the coarsest $\sigma$-algebra on $H$ that distinguishes every possible policy $\phi(\cdot,q,\cdot)$.

\begin{definition}[Counterfactual optimality]
The $P_H$,$\mathscr{M}$-counterfactual optimal policy given some cost $C:\Delta(I,V)\to \mathbb{R}$ and $P(i,v,a)$ is $\underline{\phi}^* = \argmin_\phi C(\sum_a P(v|i,a)\phi(a,i)P(a))$.
\end{definition}

\begin{definition}[Conditional optimality]
The conditionally optimal sub-history $q^*$ given some cost $C:\Delta(I,V)\to \mathbb{R}$ is $q^*=\argmin_q C(P(i,v|q))$.
\end{definition}

\begin{definition}[$\mathbf{h}_0$-optimizability]
A causal model $P_H$, $\mathscr{M}$ with $\mathbf{h}_0\in\mathcal{H}$ is $\mathbf{h}_0$-optimizable if the counterfactual and conditional optimality conditions agree:
\begin{align}
    \underline{\phi}^* &= \phi(\cdot,q^*,\cdot)\\
\end{align}
where $q^*$ is the conditionally optimal sub-history and $\underline{\phi}^*$ is the $P_H(h|\mathbf{h}_0)$, $\mathscr{M}$-counterfactually optimal policy
\begin{align}
    q^* &= \argmin_q C(\sum_a P(v|i,a,q)\phi(a,q,i)P(a|q))\\
    \underline{\phi}^* &= \argmin_{\underline{\phi}} C(\sum_a P(v|i,a,\mathbf{h}_0)\underline{\phi}(a,i)P(a|\mathbf{h}_0))
\end{align}
\end{definition}

% \begin{definition}[$\mathbf{h}_0$-optimizability]
% A causal model $P_H$, $\mathscr{M}$ with $\mathbf{h}_0\in\mathcal{H}$ is $\mathbf{h}_0$-identifiable if it is $P_\phi$-identifiable given the estimator
% \begin{align}
%     P_\phi(i,v) = \sum_a P(v|i,a,\mathbf{h}_0) \phi(a,i) P(a|\mathbf{h}_0)
% \end{align}
% \end{definition}

\begin{example}[Useless medicine]
$S\in \{0,1\}$ is information on whether or not someone is sick, $I=\{0,1\}$ is a variable representing whether or not they received treatment and $V=\{0,1\}$ is a variable representing whether or not they recovered. The experimenter is aiming (perhaps unwisely) to maximise recovery, while avoiding unnecessary treatment: $C(P(i,v))=-\mathbb{E}[V-0.5I]$. 

Suppose that the treatment does nothing, but sick people always recover anyway: $P(v|i,s) = P(v|s) = \delta_{vs}$.

The causal model consists of the distribution $P_S(s)$ and the composition of Markov kernels $\theta,\pi,\phi$ and $\mu$. The history $H$ will consist of two parts: $S$, whether or not a patient is sick and $L\in\{0,1\}$ which is whether or not the treatment decisions are being made in the ``learning'' phase. Thus $H=S\times L$, and $\mu(i,[s,l],v)=\delta_{vs}$. Define $\mathbf{h}_0\in\mathcal{H}=\{l=1\}=\{(s,l)|l=1\}$.

First, we'll consider a version of the problem that is not $\mathbf{h}_0$-identifiable. Take $A=\varnothing$, $Q=[0,1]$ and $\phi(q,i)=\text{Bernoulli}(q)$. In the learning phase, suppose $P(q|s,l=1)=\delta(q-s)$. That is treatment is given with probability $q$ and during the learning phase $\pi$ people are treated iff they are sick.

In this case, $p(v|i=1,q=1,l=1)=\delta_{v1}$ while $p(v|i=1,q=0,l=0)=\delta_{v0}$ (where conditioning on the impossible $i=1,q=0$ is defined by adding independent noise to $\phi$ and taking the limit as that noise goes to 0), so policy exchangeability does not hold. Domain stability trivially holds, however, as $A=\varnothing$.

% To see that the problem is not identifiable, note that $p(v|i=1,l=1)=\delta_{v1}$ while $p(v|i=0,l=1)=\delta_{v0}$. Thus if we were to estimate $P(i,v|q=0,l=0)$ we would calculate

% \begin{align}
%     \tilde{P}(i,v|q=0,l=0)&=P(v|i,l=1) \phi(0,i) \\
%     \tilde{P}(i,v|q=0)&=\delta_{v0}
% \end{align}

% While in fact

% \begin{align}
%     P(i,v|q=0,l=0)&=
% \end{align}

% Enumerate possible treatment policies with $Q\in \{0,1\}\times[0,1]$ where 
% $$\phi(a,[q_0,\epsilon],i) = (1-\epsilon)\left[q_0\delta_{ai} + (1-q_0)(1-\delta_{ai})\right] + \epsilon$$
% $q_0\in \{0,1\}$ indicates whether sick or healthy people are more likely to be treated and $\epsilon$ is the degree to which treatment is randomised.



% Suppose furthermore that only sick people are treated. That is, given $Q=\{0,1\}$ and $\phi$ such that $\phi(\_,q,i)=\delta_{qi}$, we have $\pi(q,s)=\delta_{qs}$. We take $A=\emptyset$

% The above assumptions lead straightforwardly to $P(v|i)=\delta_{vi}$, a typical case of ``correlation is not causation''. Recovery is perfectly matched with treatment, but this is contingent on the unwise choice of policy in which only sick people were treated. If people were treated randomly we would find that recovery was independent of treatment.

% Returning to our cost function, it would naively appear that the policy $\phi_{q1}:A\mapsto \delta_{i1}$ is optimal:

% \begin{align}
%     C(P(v|i)\delta_{i1}) &= 0.5\\
%     C(P(v|i)\delta_{i0}) &= 0
% \end{align}

% However, if we adjust for $S$ we find that we should not treat. $P(v|i,q,s)=P(v|s)$ and so $P(i,v|q,s)=P(v|s)P(i|q)$ and

% \begin{align}
%     C(P(v|s)P(i|q)) = 0.5q - s
% \end{align}

% Here, the problem was that the set $A=\emptyset$ was not large enough. If $S$ were in $\mathcal{A}$ we could have correctly concluded that sickness and not treatment was driving recovery.
\end{example}